% Copyright 2004 by Till Tantau <tantau@users.sourceforge.net>.
%
% In principle, this file can be redistributed and/or modified under
% the terms of the GNU Public License, version 2.
%
% However, this file is supposed to be a template to be modified
% for your own needs. For this reason, if you use this file as a
% template and not specifically distribute it as part of a another
% package/program, I grant the extra permission to freely copy and
% modify this file as you see fit and even to delete this copyright
% notice. 

\documentclass{beamer}
% Replace the \documentclass declaration above
% with the following two lines to typeset your 
% lecture notes as a handout:
%\documentclass{article}
%\usepackage{beamerarticle}


% There are many different themes available for Beamer. A comprehensive
% list with examples is given here:
% http://deic.uab.es/~iblanes/beamer_gallery/index_by_theme.html
% You can uncomment the themes below if you would like to use a different
% one:
%\usetheme{AnnArbor}
%\usetheme{Antibes}
\usetheme{Bergen}
%\usetheme{Berkeley}
%\usetheme{Berlin}
%\usetheme{Boadilla}
%\usetheme{boxes}
%\usetheme{CambridgeUS}
%\usetheme{Copenhagen}
%\usetheme{Darmstadt}
%\usetheme{default}
%\usetheme{Frankfurt}
%\usetheme{Goettingen}
%\usetheme{Hannover}
%\usetheme{Ilmenau}
%\usetheme{JuanLesPins}
%\usetheme{Luebeck}
%\usetheme{Madrid}
%\usetheme{Malmoe}
%\usetheme{Marburg}
%\usetheme{Montpellier}
%\usetheme{PaloAlto}
%\usetheme{Pittsburgh}
%\usetheme{Rochester}
%\usetheme{Singapore}
%\usetheme{Szeged}
%\usetheme{Warsaw}

\title{Urysohn's Lemma}



\author{Brannon Basilio}
% - Give the names in the same order as the appear in the paper.
% - Use the \inst{?} command only if the authors have different
%   affiliation.

\institute[University of Hawaii] % (optional, but mostly needed)
{
	University of Hawaii
}
% - Use the \inst command only if there are several affiliations.
% - Keep it simple, no one is interested in your street address.

\date{February 27, 2017}
% - Either use conference name or its abbreviation.
% - Not really informative to the audience, more for people (including
%   yourself) who are reading the slides online

\subject{Theoretical Computer Science}
% This is only inserted into the PDF information catalog. Can be left
% out. 

% If you have a file called "university-logo-filename.xxx", where xxx
% is a graphic format that can be processed by latex or pdflatex,
% resp., then you can add a logo as follows:

% \pgfdeclareimage[height=0.5cm]{university-logo}{university-logo-filename}
% \logo{\pgfuseimage{university-logo}}

% Delete this, if you do not want the table of contents to pop up at
% the beginning of each subsection:
\AtBeginSubsection[]
{
  \begin{frame}<beamer>{Outline}
    \tableofcontents[currentsection,currentsubsection]
  \end{frame}
}

% Let's get started
\begin{document}

\begin{frame}
  \titlepage
\end{frame}



% Section and subsections will appear in the presentation overview
% and table of contents.
\section{Definitions}

\subsection{Basic Definitions}

\begin{frame}{Basic Definitions}
  \begin{itemize}
  \item {
    \textit{A topological space $(X, \mathcal{T})$ consists of a non-empty set $X$ together with a fixed family $\mathcal{T}$ of subsets of $X$ satisfying}\\
		(T1) \quad $X, \emptyset \in \mathcal{T}$\\
		(T2) \quad \textit{the intersection of any two sets in $\mathcal{T}$ is in $\mathcal{T}$.}\\
		(T3) \quad \textit{the union of any collection sets in $\mathcal{T}$ is in $\mathcal{T}$.}\\
  \pause
  }
  \item {
    \textit{A topological space $(X, \mathcal{T})$ is normal if for every pair of disjoint non-empty closed sets $C, D \subseteq X$ there exists disjoint non-empty open sets $U, V$ such that $C \subseteq U$ and $D \subseteq V$.}
    \pause
  }
  \item {\textit{A map $f : X \rightarrow Y$ of topological spaces $(X, \mathcal{T}_X)$ and $(Y, \mathcal{T}_Y)$ is continuous if $U \in \mathcal{T}_Y \Rightarrow f^{-1}(U) \in \mathcal{T}_X$}.}
  \end{itemize}
\end{frame}

\subsection{Definitions for Urysohn's Lemma}

% You can reveal the parts of a slide one at a time
% with the \pause command:
\begin{frame}{Definitions for Urysohn's Lemma}
  \begin{itemize}
  \item {
    \textit{The dyadic rationals in $[0, 1]$ is the set $\mathbb{D} = \big\{r = \frac{m}{2^n} \vert m,n \in \mathbb{N} \text{ and } 0 \leq m \leq 2^n, n \geq 0 \big\}$.\\Note this set is dense in $[0, 1]$.}
    \pause % The slide will pause after showing the first item
  }
  \item {   
    \textit{Let $(X, \mathcal{T})$ be a topological space. A Urysohn family is a collection of open sets indexed by $\mathbb{D}$}
$\mathcal{U} = \{U_r \vert r \in \mathbb{D}\}, \text{ such that } s < t \Rightarrow \overline{U}_s \subset U_t$
  }
  % You can also specify when the content should appear
  % by using <n->:
  \item<3-> {
    \textit{The associated Urysohn function $f = f_{\mathcal{U}} : X \rightarrow [0, 1]$ is defined by}

	\[
	f(x) = 
		\begin{cases}
			\inf \{r \in \mathbb{D} \vert x \in U_r\} & \text{ if } x \in U_1,\\
			1 & \text{ otherwise.}
		\end{cases}
	\]
  }
  \end{itemize}
\end{frame}


\end{document}


