\documentclass[12pt]{article}
\usepackage[fleqn]{amsmath}
\usepackage{amssymb} %maths
\usepackage{amsmath} %maths
\usepackage[utf8]{inputenc} %useful to type directly diacritic characters
\usepackage{setspace}
\usepackage[letterpaper, portrait, margin=1in]{geometry}
\usepackage{tabu}
\usepackage{graphicx}


\newcommand{\T}{\mathcal{T}} %makes T topology
\newcommand{\D}{\mathbb{D}} %makes dyadic D
\newcommand{\U}{\mathcal{U}} %makes dyadic D




\begin{document}





\center{\subsection*{Urysohn's Lemma}}

\textit{A topological space $(X, \mathcal{T})$ is normal if and only if for every pair of disjoint closed sets $A$ and $B$ there is a continuous function $f : X \rightarrow [0, 1]$ such that $f | A = 0$ and $f | B = 1$.}




\flushleft\subsubsection*{Definition}
\textit{A topological space $(X, \mathcal{T})$ consists of a non-empty set $X$ together with a fixed family $\mathcal{T}$ of subsets of $X$ satisfying}
		\begin{align*}
			(T1)& \quad X, \emptyset \in \mathcal{T}\\
			(T2)& \quad \textit{the intersection of any two sets in $\mathcal{T}$ is in $\mathcal{T}$.}\\
			(T3)& \quad \textit{the union of any collection sets in $\mathcal{T}$ is in $\mathcal{T}$.}
		\end{align*}




\subsubsection*{Definition}
\textit{A topological space $(X, \mathcal{T})$ is normal if for every pair of disjoint non-empty closed sets $C, D \subseteq X$ there exists disjoint non-empty open sets $U, V$ such that $C \subseteq U$ and $D \subseteq V$.}\\




\subsubsection*{Definition}
\textit{The dyadic rationals in $[0, 1]$ is the set $\mathbb{D} = \big\{r = \frac{m}{2^n} \vert m,n \in \mathbb{N} \text{ and } 0 \leq m \leq 2^n, n \geq 0 \big\}$. Note this set is dense in $[0, 1]$.}




\subsubsection*{Definition}
\textit{Let $(X, \mathcal{T})$ be a topological space. A Urysohn family is a collection of open sets indexed by $\D$}
$$\mathcal{U} = \{U_r \vert r \in \D\}, \text{ such that } s < t \Rightarrow \overline{U}_s \subset U_t$$




\subsubsection*{Definition}
\textit{The associated Urysohn function $f = f_{\mathcal{U}} : X \rightarrow [0, 1]$ is defined by}

	\[
	f(x) = 
		\begin{cases}
			\inf \{r \in \D \vert x \in U_r\} & \text{ if } x \in U_1,\\
			1 & \text{ otherwise.}
		\end{cases}
	\]




\subsubsection*{Definition}
\textit{A map $f : X \rightarrow Y$ of topological spaces $(X, \T_X)$ and $(Y, \T_Y)$ is continuous if $U \in T_Y \Rightarrow f^{-1}(U) \in T_X$}.





\center\subsubsection*{Claim: Urysohn functions are continuous}
\flushleft{Let} $f$ be the Urysohn function associated to a Urysohn family $\U = \{U_r\}$ for a space $(X, \T_X)$. We shall show that for any $s \in (0, 1)$ that $f^{-1}[0, s)$ and $f^{-1}(s, 1]$ are open.\\
Notice that
	\begin{enumerate}
		\item $f(x) < r \Rightarrow x \in U_r$\\
		\item This implies $x \notin U_r \Rightarrow f(x) \geq r$\\
		\item Moreover, $x \in U_r \Rightarrow f(x) \leq r$.\\
		\item Thus, $f(x) > r \Rightarrow x \in X -  U_r$.
	\end{enumerate}
Showing that $f^{-1}[0, s) = \cup \{U_r \vert r < s, r \in \D\}$, which is open and $f^{-1}(s, 1] = \cup \{X - \overline{U}_r \vert r > s, r \in \D\}$, which is also open, prove that $f$ is continuous.





\center\subsubsection*{Proof of Urysohn's Lemma}
\flushleft
$\boxed{\Leftarrow}$ Easy; Suppose $A, B \subseteq X$ are disjoint nonempty closed sets. Let $f : X \rightarrow [0, 1]$ be a continuous functions such that $f | A = 0$ and $f | B = 1$. Thus, $A \subseteq f^{-1}([0, \frac{1}{2})$ and $B \subseteq f^{-1}((\frac{1}{2}, 1])$. Since $f$ is continuous, both preimages are open and disjoint. \\~\\

$\boxed{\Rightarrow}$ Let $A$ and $B$ be closed sets of $X$. Also, let $U_1 = X - B$. We can recursively construct a Urysohn family of open sets $\U$ such that the associated function $f$ will be our desired function. 

\begin{center}
	\includegraphics[scale = 0.9]{UrysohnsDiagram}
\end{center}












\end{document}




